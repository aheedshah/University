\chapter{Limitations and Future Works}
\section{Technical Challenges}
During the application's development, numerous technical challenges required learning many hard and soft skills. As the technologies used in the project's development were unfamiliar and the need for extensive research during the implementation stage, the development of the application was slowed down. Consequently, the MVP for employers could not be completed on time, but an MVP for job seekers and a customer-facing web application was created. The MVP for job applicants allows an applicant to log in, view potential matches, and apply for jobs with a click of a button. A React-based front end was also developed, which received positive feedback during user testing.

Due to the feedback received during the testing phase, the original proposed designs required modifications. The changes done to the designs included changing the login page, home page and others. The changes done are given in detail in the testing chapter. Additionally, time was spent adapting the designs to ensure compliance with accessibility guidelines. These changes included increasing the colour contrast ratio between elements, using descriptive alt tags, and making URLs more accessible for users with disabilities.

Building an application using React was a new experience, presenting several challenges. These included learning how to render components in React and integrating Firebase with React. Also, handling front-end requests and back-end responses, such as on the login page, was challenging. Learning these technologies and optimising their performance took significant time, but the effort paid off in the end.

Unit testing was done for every component to ensure the high quality of the product and prevent errors. Unit tests were run each time the application was changed or committed to the version control, and until all of them didn't pass, the changes did not commit. Adding unit tests using technologies like Jest was also something which proved to be challenging and took me some time to familiarise myself with. But, at the end of the implementation stage, I tested each component, and it worked well in the end. 

In summary, the application's development presented numerous technical challenges, including learning new technologies and making the product compliant with accessibility guidelines. However, despite these challenges, an MVP for job applicants and a customer-facing React front end was created.

\newpage
\section{Evaluation}
The development of this web app didn't go as smoothly as planned. The development of this web app started with the unfamiliarity with the technologies meant to be used to build this application. While learning new technologies was ultimately valuable, it slowed development and prevented the MVP from being completed on schedule.

Furthermore, managing a part-time job and studying challenging coursework simultaneously created a tight schedule that made it difficult for me to dedicate the necessary time required to fully develop this web application to its intended state.

Conducting user testing posed another challenge since the application was only accessible on my local machine, making it necessary for testers to use my machine for testing and providing feedback. Coordinating with the testers proved challenging, as finding a mutually convenient day and time for so many testers was difficult. Consequently, a limited number of testers could evaluate the application and provide feedback when more testers should have participated.

Due to the delayed start in the development of this application caused by user testing and design iteration, the unit testing phase began later in the development process as planned, requiring additional work at the last stages of the implementation. User testing of the final web app and designs could also not be completed because of the late completion of the implementation of the final web app. 

Overall, while an MVP was created for an applicant, the development of this web app was hindered by various challenges, including unfamiliarity with technologies, time constraints, and difficulties coordinating user testing. Many of these things could've been handled better and should be changed going forward in developing this application.

\section{Future Works}
I firmly believe that Job Crop has the potential to be highly beneficial to job seekers and employers alike. However, it is worth noting that the application's current iteration is merely a starting point, which will be expanded further, making it a more sophisticated and comprehensive system.

Many things need to be developed and changed to achieve this, such as the Algorithm used. While the current algorithm only returns true or false based on questions, I intend to build a more intricate algorithm that considers additional factors, such as the number of technology matches between the applicant and the job.

A separate user interface needs to be developed for employers. This interface should allow them to add, edit, and analyse their job postings. To create an efficient app for employers, gathering input from HR professionals in different organizations is essential. Once jobs are posted, they should appear on the applicant's side if they match their skills and experience.

In conclusion, with the addition of these features and others, Job Crop has the potential to become a strong competitor against traditional job search applications, providing a better experience for both job seekers and employers.