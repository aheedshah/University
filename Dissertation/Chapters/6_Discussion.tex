\chapter{Discussion}
The initial objective of this project was to create a web application for people in the tech industry to look at a job posting and determine if they want to apply without reading multiple paragraphs in the job description, like in traditional job search applications. The background and research chapter emphasise the need to create a job-searching application with the above features and specifications in the current market, which will benefit both Employers and Applicants.

During the implementation stage, an MVP was created for applicants, which shows how an applicant would log in and be shown a potential match. The MVP shows, to some extent, what the proposed application would look like. While the web app still needs some development before a full MVP can be created that a potential applicant and an employer can use, a customer-facing experience of the application is implemented, which shows the new experience and features this new application could bring into the market of the job-searching industry and how an application such as Job Crop would benefit both Employers and Applicants in the tech industry.

To apply for a job, applicants can log in, complete a questionnaire, browse through different job matches, and select the one that aligns with their preferences. Employers can post and edit job listings and match them with potential applicants. Due to the time shortage for the development of this app and the need to learn new technologies, the latter could not be built in time for this project. Moving forward, this needs to be built to reach the MVP for this application. 

During user testing, participants found it easy to navigate the web app, log in, and view potential job matches. They could quickly find buttons to skip or apply to jobs without assistance and expressed positive feedback on the app's overall design and concept. This successful user testing was a significant milestone towards achieving the project's goals.

Ensuring the accessibility of the web app was a top priority during the implementation phase. The app's design and functionality were optimised from the initial designs to cater to users with disabilities or impairments. The web app's content was made easily navigable with keyboard-only controls. Additionally, the app's buttons were designed to be big and clear, and the colour ratios were set to comply with W3C Accessibility Standards, further enhancing the accessibility of the app \parencite{Reference42}.

While the current application shows what a customer-facing web app would look like and the process a potential applicant would follow, it requires significant development to meet current market standards and attain the MVP. Notably, a user interface needs to be designed for employers to add, edit and analyse their job postings. Additionally, questionnaires must be created for employers and applicants to specify job details which will be used in the matching algorithm. Although this list is not comprehensive, accomplishing these tasks is crucial for achieving the application's MVP.