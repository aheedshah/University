% Chapter 1

\chapter{Introduction} % Main chapter title

\label{Chapter1} % For referencing the chapter elsewhere, use \ref{Chapter1} 

%----------------------------------------------------------------------------------------

% Define some commands to keep the formatting separated from the content 
\newcommand{\keyword}[1]{\textbf{#1}}
\newcommand{\tabhead}[1]{\textbf{#1}}
\newcommand{\code}[1]{\texttt{#1}}
\newcommand{\file}[1]{\texttt{\bfseries#1}}
\newcommand{\option}[1]{\texttt{\itshape#1}}

%----------------------------------------------------------------------------------------

\section{Overview}
The main goal of this project is to make a Job Search Engine specifically for people in the tech industry and provide them with relevant roles only based on their skill set and the company's demands.

\section{Aim and Motivation}
Searching for jobs in the tech industry can be a nightmare. While searching for jobs in the industry, I realised that job searching is still stuck in the dark ages. Going through the entire description of the job just to find out that the tech stack does not match what I want to do was infuriating. Not replying to applicants has become a norm \parencite{Reference6}. As an applicant, it was frustrating not to get a reply after spending hours creating a resume and answering job-specific questions. In addition, many jobs on these search engines were outdated.

The idea to re-think applying came from that. The idea of Job Crop is to crop out all the outdated jobs and hard-to-read paragraphs of a job and give users an all-new design of a job search engine which helps them find exactly the job they want without having to scroll endlessly through countless pages to find it. 

I plan to build a job-searching web app for people in the tech industry to look at a job posting and determine if they want to apply without reading multiple paragraphs in the job description. 

\section{Resources}
This web application will primarily be written in React on the Front-end, and database services like Firebase will also be needed while building this application to store user details. Firebase was chosen because it has a generous free tier, allowing the MVP to be built free of cost. 

Other technologies/languages, like React Testing Library or Jest, will also be used for testing purposes.

The primary tool required for this application will be IntelliJ IDEA, as it is an all-in-one solution to all the different technologies during this application's development process. 

Information about these tools and others are written in detail in Chapter \ref{Chapter2}.

\section{Report Plan}
This report will be built on during the following months as follows:
\begin{itemize}
    \item \textbf{Chapter 2, Background Research:} This chapter includes the background research done before building and designing this web app. This chapter includes research done on other competitors in the market and the pros and cons of each of them. An overview of why this project is worth doing and the motivation behind it is also in this chapter.
    \item \textbf{Chapter 3, Design:} This chapter includes the design of each app component. It will also reason why one design was chosen over the other and include the prototypes for each design based on technical and functional requirements.
    \item \textbf{Chapter 4, Implementation:} This chapter includes the implementation of each app component and will explain how they were built. It will include a database schema and code snippets wherever required.
    \item \textbf{Chapter 5, Testing:} This chapter includes the testing done during the development stage of the software. This will include technical testing, user testing, and other testing methods which will help identify the success of the software.
    \item \textbf{Chapter 6, Discussion:} This chapter includes analysing what is built into this project. This chapter relates the product built to the background research and introduction section.
    \item \textbf{Chapter 7, Limitations and Future Works:} This chapter includes the next steps for this project and the limitations of building this application. This will also show how this application can be further developed and the key areas that must be developed to make it work.
    \item \textbf{Chapter 8, Conclusion:} This final chapter will include the final thoughts of the web app based on success determined by the testing stage of the software. This will also include a critical evaluation of the web app to see where it needs future advancements.
\end{itemize}

\section{Software Plan}
As this project is quite huge, building the whole web application is significantly beyond the scope of a project. So, at the end of the development stage of this project, an MVP should be created where an applicant can access this web app, log in and see potential matches. The web app should be developed based on the testing and designs done beforehand and look similar to those at the end. All the new features will be rigorously tested with users as they are built to analyse the audience's behaviour when interacting with the product.